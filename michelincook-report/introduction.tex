\section{Introduction} 
The kitchen plays an important emotional role in our lives. Everyone enjoys getting together with friends or family for meals, especially during festivals or holidays. These celebrations are usually potluck in nature and it would be awkward to bring in off the shelf food for such meet-ups. Moreover, it is believed widely that ''The easiest way to Man's heart is through his stomach'', making it more important to cook for you loved ones. It also becomes essential to cook your own meal when you are on a particular diet or want to have a healthier lifestyle. So it is no exaggeration to say that ability to cook improves your life greatly.

But there are many people who do not take to cooking because of the fear that manifests from this exercise. Some people are overwhelmed by the recipes which may be complicated to cook from or overly long and fear missing the steps, some are intimidated by the cooking techniques involved because of not clearly understanding them, some fear that they may overcook the food, some fear seasoning the food wrongly and there are some others who worry about how the food looks. This phobia or fear of cooking is termed as Mageirocophobia.

However we believe that it is possible to alleviate the above mentioned fear by providing an easy and intuitive user interface to deliver the cooking instructions.

\section{Problem Statement}
The current available system for delivery of cooking instructions is difficult to follow and confusing to cook from. We did a survey to find out what are the different medias that people use to cook from and found out that people use recipe blogs, YouTube videos, paperback recipe books, recipe apps etc.., 

There are several shortcomings of using the above listed medias for cooking. The conventional paperback and recipe blogs are difficult to cook from because of them being static and less intuitive. Using YouTube videos during cooking is cumbersome as it is difficult when trying to refer to a previous step and results in a lot of pause, rewind and forward. The current available apps provides solution to many of the above mentioned problems but we believe that the cooking experience can be further enhanced by feature set of MichelinCook.

Our mobile application mainly aims at providing the user with cooking instructions in the form of self-timed digital cards \cite{CardView}. Flash Cards have been used as an effective strategy for studying, we use a modified version of the flash cards to make up the cards used to provide the cooking directions. The cooking procedure is broken down into many steps and each card presents an individual step with a timer(which runs down from the required time for that step). The important steps are additionally presented with photographs on the cards and can be used as references to further enhance the cooking process.