


\section{Evaluation}

We used Lean Startup \cite{LeanStartup} model for evaluation of our mobile application idea. The advantage of using Lean Startup Methodology in mobile development includes,
\begin{itemize}
	\item Faster time to market
	\item Lower Cost Overall and Upfront Cost
	\item Less Financial Risk
	\item Build a Better product that customers want.
	\item Less wasted time
\end{itemize}

In our case as this is a course project, the most relevant use of Lean Startup Methodology is for building a better product that customer wants and with a shorter design cycle time.

\subsection{Building an MVP}
A minimum viable product \cite{MVP} is a product with only the core features. We plan to use the current version of  our application as MVP for the purpose of testing. The MVP will be deployed for testing among early adopters whose feedback will be collected for further enhancement of the application.

The MVP is a way to test with the objective of answering four questions:
\begin{itemize}
	\item Do users recognize that they have the problem we are trying to solve?
	\item If there is a solution, would they use it?
	\item Would they use our app?
	\item Are we building a solution for the above problem?
\end{itemize}

If the answer is yes for four of the above questions, then we have an actual requirement for such an application. Building an MVP is a way to collect feedback, user response, and test the waters. The challenge of building an MVP can be a bit more difficult on mobile. There is a higher bar for mobile applications. We have to be a bit more refined as people are less patient on mobile and there is a much lower tolerance for bugs.


\subsection{User Feedback}
Each feature built are like experiments to see how users adopt and engage in the product. In the early stage of building a product, it is extremely important to talk to your customers. The early adopters, beta testers, and evangelists are key for customer discovery. We need to evaluate if our core proposed product features solve their problem. Data gathered from the first group of beta users will be key for figuring out what features to focus on.

\subsection{Quantitative and Qualitative Metrics}
The feedback from users are extremely useful and comes in both qualitative and quantitative forms. There are certain key metrics to focus on that will demonstrate health. One of the most important focus for our application is to measure engagement. Nothing signals stickiness and that you have something with your MVP than to have good engagement in your app. Some hard metrics that shows engagement in mobile includes:
\begin{itemize}
	\item Active Users : Number of active users a day/week/month is essential for measuring
	\item Session Length : How long do the users spend in your app. The time they open to the time its closed.
	\item Session Interval : How long in between each session of usage
	\item Screen Flow : Interactions in the app between each screens, duration on each screens, and the total number of occurrences in each screen.
	\item Retention Rate : the percentage of users who return to the app based on the date of their first visit.
\end{itemize}

In addition to these quantitative forms of measuring engagement, it is also important to observe and watch users engage in our app for UX purposes.  Direct communication from users may be best for clear feedback of the app.

Any user data points gathered, both qualitative and quantitative, can be extremely beneficial  for a glance at user behavior to learn and provide direction for app development. It is crucial to track user behavior, feedback, and use the information when building new features.

\section{Future Work}

In this section, we discuss our future work from following perspectives: implementation, functionality and user experience.

\subsection{Implementation}
Currently we only implement ''MichelinCook'' on Android system. In the future, we also plan to implement it on iOS, which is also one of the most popular smartphone system. And we also plan to upload our app to both Android Market and iOS App Market.

\subsection{Functionality}
Self-timed digital card is very helpful for cooking. However, it needs users to manually pause/resume the cards. In the future, we may add voice control so that users can cook the dish without ever having to click the app.

In addition, we plan to add functions to allow the user to upload their own recipes and some happen-before  relations (e.g.,  step 1 should go first than another step 2) for  their recipes. Our app can then automatically generate the optimized step sequence with corresponding time card.

\subsection{User experience}
We plan to make a survey of our app among other students and ask some feedback about the current interface, functions and features. Using their feedback we can add more new functions and features.